\documentclass[a4paper]{article}

\usepackage[french]{babel}
\usepackage[T1]{fontenc}
\usepackage[utf8]{inputenc}
\usepackage{amsmath}
\usepackage{graphicx}
\usepackage[colorinlistoftodos]{todonotes}

\title{Série 26}

\author{Florian Gatignon}

\date{Mercredi 20 mai 2015}

\begin{document}
\maketitle

\begin{abstract}

Aujourd’hui, comme la dernière fois, Florian Gatignon, notre jeune
aventurier à la recherche d’exercices qu’il serait capable de faire. Il commença donc par créer un résumé tout pourri pour faire semblant qu’il avait fait cette série sérieusement en ne commençant pas à mettre des
images ridicules dès la première page. Après 24 séries absolument mal
présentées, faire des séries de manière humoristique en incluant une présentation irréprochable pour faire semblant d’avoir bien travaillé, était-ce
vraiment le seul moyen que Florian avait trouvé pour espérer avoir des
notes approximativement similaires à la moyenne? Ne pourrait-il pas, par
exemple, faire ses séries correctement et les remettre aux propres et avoir
une moyenne de 75\%, comme en deuxième année?
Non. Car la vérité était que de nombreux obstacles se dressaient devant
lui, à commencer par son nouvel assistant, qui ne confondait pas ses 0
avec des 6.\newline
Un jour ou l’autre, il lui aurait fallu s’en débarrasser.
\end{abstract}

\subsubsection*{Exercice 1}

\renewcommand{\labelenumi}{\alph{enumi}) }
\begin{enumerate}
\item
        \begin{itemize}
		\item        \textbf{$n = 1$}
                $$\frac{1³}{3}  = \frac{1}{3}  \le \sum\limits{n=1}^1 $$
        \end{itemize}

\end{enumerate}

\end{document}